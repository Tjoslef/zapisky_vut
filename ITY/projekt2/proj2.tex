\documentclass[twocolumn, 11pt, a4paper]{article}
\usepackage[utf8]{inputenc}
\usepackage[T1]{fontenc}
\usepackage{hyperref}
\usepackage{mathtools, amssymb, amsthm} % imports amsmath
\usepackage{lmodern}
\usepackage[czech]{babel}
\hbadness = 10000
\usepackage[textwidth=18.6cm, textheight=26.0cm]{geometry}
\DisableGenericHook{cmd/deferred@thm@head}
\newtheorem{definice}{Definice}
\newtheorem{veta}{Veta}
\begin{document}
\begin{titlepage}
	\begin{center}
		{\Huge \textsc{Vysoké učení technické v Brně}} \\[0.5em]
		{\huge \textsc{Fakulta informačních technologií}} \\[0.6em]
		\vspace{\stretch{0.382}}
		\quad\hbox{\huge{Typografie a publikování -- 2. projekt}} \\[0.6em]
		\quad\hbox{\huge{Sazba dokumentů a matematických výrazů}}
		\vspace{\stretch{0.618}}
	\end{center}
	\begin{flushright}
		{\Large 2025} \hfill {\Large Josef Pasek (xpasekj)}
	\end{flushright}
\end{titlepage}
\twocolumn
\section*{Úvod}
V této úloze vysázíme titulní stranu a ukázku matematického textu,
v němž se vyskytují například rovnice ~\eqref{equation:1:1} na straně~\pageref{equation:1:2}, Věta ~\ref{veta.1} nebo Definice ~[\ref{definice.2}].
Pro vytvoření těchto odkazů používáme kombinace příkazů
\texttt{\textbackslash label}, \texttt{\textbackslash ref}, \texttt{\textbackslash eqref} a \texttt{\textbackslash pageref}
Před odkazy patří nezlomitelná mezera.
Text zvýrazníme pomocí příkazu \texttt{\textbackslash \emph}, strojopisné písmo pomocí \texttt{\textbackslash texttt}.
Pro LaTeXové příkazy (s obráceným lomítkem) použijeme \texttt{\textbackslash verb}.

Titulní strana je vysázena prostředím titlepage a nadpis je v optickém středu
s využitím zlatého řezu, který byl probrán na přednášce.
Na titulní straně jsou tři různé velikosti písma a mezi dvojicemi řádků textu
je řádkování se zadanou  velikostí 0,5 em a 0,6 emi\footnote{Použijte správnou velikost mezery mezi číslem a jednotkou}.

\section{Matematický text}
Symboly číselných množin sázíme makrem \texttt{\textbackslash mathbb},
kaligrafická písmena  makrem \texttt{\textbackslash mathcal}.
Pozor na tvar i sklon řeckých písmen: srovnejte \texttt{\textbackslash varrho} a \texttt{\textbackslash varvarrho}.Kon\-
strukce\hbox{ \verb|{${}$}| nebo \texttt{\textbackslash hmbox\{\}} zabrání zalomení výrazu.}

Pro definice a věty slouží prostředí definovaná příkazem \texttt{\textbackslash newtheorem} z balíku amsthm.
Tato prostředí obracejí význam \texttt{\textbackslash emph}:
uvnitř textu sázeného kurzívou se zvýrazňuje písmem v základním řezu.
Důkazy se někdy ukončují značkou \texttt{\textbackslash qed}.
\subsection{Pseudometrický prostor}
Pro zarovnání rovností a nerovnosti pod sebe použijte vhodné prostředí.
\begin{definice}
V pseudometrickém prostoru $\mathcal{M} = (M,\varrho)$ značí $M$ množinu bodů,
$\varrho: M \times M \rightarrow \mathbb{R}$ je zobrazení zvané pseudometrika, které pro každé body $x,y,z \in M$
splňuje následující podmínky:
\setcounter{equation}{0}
\begin{eqnarray}
    \varrho(x,x) = 0 \label{equation:2:1} \\
    \varrho(x,y) = \varrho(y,x)\label{equation:2:2} \\
    \varrho(x,y) + \varrho(y,z) \geq \varrho(x,z) \label{equation:2:3}
\end{eqnarray}
\end{definice}
\subsection{Metrika}
Funkční hodnota pseudometriky $\varrho$ se nazývá \textit{vzdá\-lenost}. Vzdálenost každých dvou bodů je nezáporná.
\begin{veta}\label{veta.1}
    Pro každé dva body $x,y \in M$ pseudometrického prostoru $(M,\varrho)\text{ platí } \varrho(x,y) \geq 0$.
\end{veta}
Důkaz: Nechť $x,y \in M \text{ a označme d} = \varrho(x,y)$ \hbox{Využitím}~\eqref{equation:2:2}  mame 2d$ = \varrho(x,y) + \varrho(y,x)$
,z nerovnosti ~\eqref{equation:2:3} 2d $\geq \varrho(x,z)$ a z rovnosti~\ref{equation:2:1} dostaneme $ \text{2d } \geq \varrho(x,y) = 0$. Odtud plyne d $\geq0$ .

Speciálním případem pseudometrických prostorů jsou prostory metrické,
v nichž dva různé body mají vždy kladnou vzdálenost.
\begin{definice}\label{definice.2}
    Nechť $\mathcal{M} = (M,\varrho)$ pseudometrický prostor, v němž platí $\varrho(x,y) \ge 0$, kdykoliv $x \neq y$.
    Potom $\mathcal{M}$ se nazývá \normalfont{metrický prostor a } $\varrho$ \normalfont{ je jeho metrika.}
\end{definice}
\section{Rovnice}
Velikost závorek a svislých čar je potřeba přizpůsobit jejich obsahu.
K tomu jsou určeny modifikátory \texttt{\textbackslash left} a \texttt{\textbackslash right}.
\begin{eqnarray}
    \lim_{p \to \infty} \left( \frac{1}{n} \sum_{i=1}^n (x_i)^p \right)^\frac{1}{p} = \left( \prod_{i=1}^n x_i \right)^\frac{1}{n}
\end{eqnarray}
Zde vidíme, jak se vysází proměnná určující limitu v běžném textu: $\lim_{m \to \infty} f(m)$.
Podobně je to i s dalšími symboly jako $\bigcup_{N \in \mathcal{M}}$ či $\sum_{i=1}^m x_{i}^2$.
S vynucením méně úsporné sazby příkazem \texttt{\textbackslash limits} budou vzorce vysázeny v podobě $\lim\limits_{m \to \infty} f(m) \text{ a }\sum\limits_{i=1}^m x_{i}^2$.
Složitější matematické formule sázíme mimo plynulý text pomocí prostředí \texttt{displaymath}.
\begin{eqnarray}
\lim_{n \to \infty} \left( 1 + \frac{x}{n} \right)^n = \sum_{n=0}^\infty \frac{x^n}{n!} \\
\sum_{\emptyset \neq X \subset P} (-1)^{|X| -1} |\bigcap X| = |\bigcup P| \\ \label{equation:1:1} \label{equation:1:2}
-\int_a^b f(x)dx = \int_b^a f(y)dy
\end{eqnarray}
Nezapomeňte rovnice, na které se odkazujete, označit vhodným jménem pomocí \texttt{\textbackslash label}.
\section{Matice}
Pro sázení matic se používá prostředí array a závorky s výškou nastavenou pomocí
\texttt{\textbackslash left, \textbackslash right}.
\[
D = \left|
\begin{array}{cccc}
  a_{11} & a_{12} & \cdots & a_{1n} \\
  a_{21} & a_{22} & \cdots & a_{2n} \\
  \vdots & \vdots & \ddots & \vdots \\
  a_{m1} & a_{m2} & \cdots & a_{mn}
\end{array}
\right| = \left|
\begin{array}{cc}
  x & y \\
  t & w
\end{array}
\right| = xw - yt
\]
Prostředí \texttt{array} lze úspěšně využít i jinde, například
na pravé straně následující definiční rovnosti.
\[
B_n =
\begin{cases}
  1 & \text{pro } n = 0 \\
  \sum\limits_{k=0}^{n-1} \binom{n}{k} B_k & \text{pro } n \geq 1
\end{cases}
\]
Jestliže sázíme jen levou složenou závorku, pak za párovým \texttt{\textbackslash right}
místo závorky píšeme tečku.
\end{document}
