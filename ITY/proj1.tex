\documentclass[twocolumn, 10pt, a4paper]{article}
\usepackage[utf8]{inputenc}
\usepackage[T1]{fontenc}
\usepackage[czech]{babel}
\usepackage{ulem,alltt,csquotes,xcolor,enumitem,hyperref}
\usepackage[left=1.8cm, top=1.8cm, right=1.8cm, bottom=1.8cm, textwidth=18.2cm, textheight=24.1cm]{geometry}
\title{Typografie a publikovani -- 1.projekt}
\author{Josef Pasek}
\author{Josef Pasek \\ \texttt{xpasekj00@stud.fit.vut.cz}}
\begin{document}
\maketitle
\section{Hladka Sazba}
Hladká sazba používá jeden stupeň, druh a řez písma. Sází
se na stránku s pevně stanovenou šířkou. Skládá se z od\-
stavců. Odstavec končí východovou řádkou. Věty nesmějí
začínat číslicí.

Zvýraznění barvou, podtržením, ani změnou písma se
v odstavcích nepoužívá. Hladká sazba je určena především
pro delší texty, jako je beletrie. Porušení konzistence sazby
působí v textu rušivě a unavuje čtenářův zrak.
\section{Smíšená sazba}
 Smíšená sazba má o něco volnější pravidla. Klasická hladká
sazba se doplňuje o další řezy písma pro zvýraznění důle\-
žitých pojmů. Existuje \enquote{pravidlo}:
\begin{quotation}
    Čím více \textsc{druhů}, \textit{řezů }, {\tiny velikostí}, \textcolor[rgb]{0, 1, 0.098}{barev}
    a jiných \texttt{efektů} použijeme, \uline{tím profesionálněji}
bude {\fontfamily{pzc}\selectfont{dokument} }vypadat. Čtenář tím bude {\Large \textbf{vždy
    nadšen!}}
\end{quotation}
\textsc{Tímto pravidlem se nesmíte \textbf{\uline{nikdy} řídit}}.Pří\-
liš časté zvýrazňování textových elementů a změny{\tiny velikostí}
písma jsou známkou \textbf{amatérismu autora} a působí \texttt{velmi
rušivě.}Dobře navržený dokument je
\uline{decentní, ne chaotický}.

Důležitým znakem správně vysázeného dokumentu je
konzistence--například \textbf{tučný řez} písma bude vyhrazen
pouze pro klíčová slova, \textit{kurzíva} pouze pro doposud ne\-
známé pojmy a nebude se to míchat. Kurzíva nepůsobí tak
rušivě a používá se častěji. V \LaTeX u ji sázíme raději pří\-
kazem \verb|\emph{text}| než \verb|\textit{text}|.

Smíšená sazba se nejčastěji používá pro sazbu vědeckých
článků a technických zpráv. U delších dokumentů vědec\-
kého či technického charakteru je zvykem vysvětlit význam
různých typů zvýraznění v úvodní kapitole.
\section{Další rady:}
\begin{itemize}[left=0pt, label=•]
    \item Nadpis nesmí končit dvojtečkou a nesmí obsahovat od\-
        kazy na obrázky, citace, poznámky pod čarou, \ldots
    \item Nadpisy, číslování a odkazy na číslované elementy musí
        být sázeny příkazy k tomu určenými. Číslování sekcí
        tohoto dokumentu je zajištěno příkazem \verb|\section|.
\pagebreak
    \item
        Poznámky pod čarou\footnotemark~používejte opravdu střídmě.
        (Šetřete i s textem v závorkách.)
    \item
        Bezchybným pravopisem a sazbou dáváme najevo úctu
        ke čtenáři. Odbytý text s chybami bude čtenář právem
        považovat za nedůvěryhodný.
    \item
        Výčet (v \LaTeX sázíme např. pomocí \texttt{itemize}) ani
        obrázek nesmí začínat hned pod nadpisem a nesmí
        tvořit celou kapitolu.
    \item
        Nepoužívejte velké množství malých obrázků. Zvažte,
        zda je nelze seskupit
\end{itemize}
\section{České odlišnosti}
    Česká sazba se oproti okolnímu světu v některých aspek\-
    tech mírně liší. Jednou z odlišností je sazba uvozovek. Uvo\-
    zovky se v češtině používají převážně pro zobrazení přímé
    řeči, zvýraznění přezdívek a ironie. V češtině se používají
    uvozovky typu \enquote{9966} místo “anglických uvozovek”. Lze je
    sázet připravenými příkazy nebo při použití UTF-8 kódo\-
    vání i přímo.

    Ve smíšené sazbě se řez uvozovek řídí řezem prvního uvo\-
    zovaného slova. Pokud je uvozována celá věta, sází se kon\-
    cová tečka před uvozovku, pokud se uvozuje slovo nebo část
    věty, sází se tečka za uvozovku.

    Druhou odlišností je pravidlo pro sázení konců řádků.
    V české sazbě do bloku by řádek neměl končit osamocenou
    jednopísmennou předložkou nebo spojkou. \mbox{Spojkou}
    \enquote{a} končit může pouze při sazbě do šířky 25 liter. Abychom
    \LaTeX u zabránili v sázení osamocených předložek, spoju\-
    jeme je s následujícím slovemi \textit{nezlomitelnou mezerou} Tu
    sázíme pomocí znaku \verb|~| (vlnka, tilda). Pro systematické
    doplnění vlnek slouží volně šiřitelný program vlna od pana
    Olšáka\footnotemark, který je vhodné si v rámci projektu vyzkoušet.

    Balíček \texttt{fontenc} slouží ke korektnímu kódovaní znaků
    s diakritikou, aby bylo možno v textu vyhledávat a kopí\-
    rovat z něj.
\section{Závěr}
Tento dokument schválně obsahuje několik typografických
prohřešků. Sekce 2 a 3 obsahují typografické chyby. V kon\-
textu celého textu je jistě snadno najdete. Je dobré znát
možnosti LATEXu, ale je také nutné vědět, kdy je nepoužít.
{\footnotetext[1]{Příliš mnoho poznámek pod čarou čtenáře zbytečně rozptyluje.}}
\footnotetext[2]{\href{http://petr.olsak.net/ftp/olsak/vlna/}{Viz http://petr.olsak.net/ftp/olsak/vlna/}}
\end{document}
