\documentclass[11pt, a4paper]{article}
\usepackage[T1]{fontenc}
\usepackage[utf8]{inputenc}
\usepackage{mathptmx}
\usepackage[czech]{babel}
\usepackage[style=ddmmyyyy]{datetime2}
\usepackage[left=2cm, top=3cm , textwidth=17cm, textheight=24cm]{geometry}
\begin{document}
\makeatletter
\begin{titlepage}
	\begin{center}
		{\Huge \textsc{Vysoké učení technické v Brně}} \\[0.5em]
		{\huge \textsc{Fakulta informačních technologií}} \\[0.6em]
		\vspace{\stretch{0.382}}
		\quad\hbox{\LARGE{Typografie a publikování -- 3. projekt}} \\[0.6em]
		\quad\hbox{\Huge{Tabulky a obrázky}}
		\vspace{\stretch{0.618}}
	\end{center}
	\begin{flushright}
		{\Large \today} \hfill {\Large Josef Pasek}
	\end{flushright}
\end{titlepage}
\makeatother
\section{Úvodni strana}
Název práce umístěte do zlatého řezu a nezapomeňte uvést \textit{dnešní} (today) datum a vaše jméno a příjmení.
\section{Tabulky}
Pro sázení tabulek můžeme použít buď prostředí \texttt{tabbing nebo prostředí} \texttt{tabular}.
\subsection{Prostředí tabbing}
Při použití tabbing vypadá tabulka následovně:
\begin{tabbing}
	Ovoce \hspace{1.5cm} \= Množství \hspace{0.5cm} \= Jednotka \hspace{0.5cm} \= Cena za jedn. \hspace{0.5cm} \= Cena celková \kill
	\textbf{Ovoce} \> \textbf{Množství} \> \textbf{Jednotka} \> \textbf{Cena za jedn.} \> \textbf{Cena celková} \\
	Jablka \> 3 \> kg \> 25,90 Kč \> 77,70 Kč \\
	Hrušky \> 2,5 \> kg \> 27,40 Kč \> 68,50 Kč \\
	Vodní melouny \> 1 \> kus \> 35,--Kč \> 35,--Kč \\
\end{tabbing}
Toto prostředí se dá také použít pro sázení algoritmů, ovšem vhodnější je použít prostředí algorithm nebo
algorithm2e (viz sekce 3).
\subsection{Prostředí tabular}
Další možností, jak vytvořit tabulku, je použít prostředí tabular. Tabulky pak budou vypadat takto1 :
\begin{tabular}{sloupce}
\end{tabular}
\end{document}
