\documentclass{article}
\usepackage[utf8]{inputenc} % Specify UTF-8 encoding
\usepackage{mathtools, amssymb, amsthm} % imports amsmath
\usepackage{listings} % for code snippets
\usepackage{enumitem} % for better list control
\usepackage[a4paper, margin=1in]{geometry} % adjust page margins
\usepackage{xcolor} % for \textcolor

\sloppy % allow LaTeX to relax overfull box constraints

\lstset{
    basicstyle=\ttfamily, % Use a monospaced font
    breaklines=true,      % Enable line breaking
    frame=single,         % Frame the code
    postbreak=\mbox{\textcolor{red}{$\hookrightarrow$}\space}, % Nice line-break marker
}

\begin{document}

\section*{Příkazy skoku:}
\begin{itemize}[noitemsep] % Reduce space between list items
    \item \textbf{break} - Vyskočí z jednoho cyklu.
    \item \textbf{continue} - Vrátí se na podmínku.
    \item \textbf{goto} -
    \item \textbf{return} - Výhoda oproti break je, že se dá použít ve vícei
        zanořených cyklech a vyskočí ze všech.
\end{itemize}

\section*{Datový typ ukazatel, alokace paměti:}

\subsection*{Alokace paměti:}
Virtuální paměťový prostor je $2^{64}$. Do UPP se zapíše kód a statické proměnné,
které jsou pouze ke čtení, a dále heap a stack (zásobník).

\begin{itemize}[noitemsep]
    \item \textbf{Heap} - Dynamická alokace paměti.
    \item \textbf{Stack} - Paměť omezena velikostí, ukládá lokální proměnné.
\end{itemize}

Příklad alokace paměti:
\begin{lstlisting}
int example = 51;  // Zapisuje se do stacku
\end{lstlisting}

\subsection*{Ukazatel:}
Ukazatel může ukazovat kamkoliv, ale musíme si dát pozor, aby neukazoval na neplatnou paměť.

Příklad práce s ukazatelem:
\begin{lstlisting}
\int *ptr = &example;  // Uloží adresu hodnoty example
\*ptr = 51;            // Dereference, získá hodnotu na adrese
ptr = 0;              // Ukazuje na adresu 0, NULL
\end{lstlisting}

Bazový typ ukazatele určuje velikost paměti, kterou ukazuje:
\begin{itemize}[noitemsep]
    \item \textbf{\&} - Vrací adresu proměnné.
    \item \textbf{*} - Dereference, vrací hodnotu na adrese.
\end{itemize}

\subsection*{Dynamická alokace paměti:}
Na heapu je nutné použít ukazatele, protože nelze přímo vytvářet pojmenované
proměnné. Programátor je zodpovědný za správnou alokaci i dealokaci paměti.

Příklad dynamické alokace:
\begin{lstlisting}
int \*pi = malloc(sizeof(int));  // Alokuje paměť na heapu
if (pi == NULL) {
    return ERR_MALLOC;  // Ověří, zda malloc uspěl
}
free(pi);  // Uvolní paměť
\end{lstlisting}

Musíme dávat pozor na ztrátu ukazatele, jinak vzniká \textbf{memory leak}.

\subsection*{Pole:}
Pole je datový typ, který ukládá prvky stejného typu. V C není kontrola mezí
polí, což může vést k \textbf{buffer-overflow} chybám.

\end{document}

