\documentclass{article}
\usepackage{mathtools, amssymb, amsthm} % imports amsmath

\begin{document}
\sloppy

\section{Opakování pointerů}

Charakterový ukazatel \texttt{char *p = array} můžeme používat k přístupu na prvky pole pomocí pointerové aritmetiky.
\texttt{*p} odkazuje na 0. index, zatímco \texttt{*(p+1)} odkazuje na 1. index (jiný zápis pro \texttt{p[1]}).
Například, \texttt{char str[5] = "abc"} vytvoří řetězec, který je ukončen znakem \texttt{'\textbackslash0'}.

Více rozměrná pole můžeme deklarovat například jako \texttt{float matrix[3][2]}, což vytvoří matici s 3 řádky a 2 sloupci.
Pro procházení touto maticí lze použít cykly:

\texttt{for (řádky)} a \texttt{for (sloupce)}.

Dynamickou alokaci matic lze provést pomocí ukazatelů. Alokujeme paměť pro matici takto:

\texttt{float *matrix;}
a \texttt{matrix = malloc(r*s*sizeof(float));}, kde \texttt{sizeof} operátor vrací velikost datového typu.

Další možnost je dvouúrovňová alokace pro dynamickou matici:

\texttt{float **matrix;} a \texttt{matrix = malloc(r * sizeof(float *));},
kde následně pro každý řádek alokujeme paměť pro sloupce: \texttt{*(matrix + r) = malloc(s * sizeof(float));}.

\section{Funkce}

V kontextu pointerů můžeme předávat adresy proměnných do funkcí, abychom mohli měnit jejich hodnoty. Například:

\texttt{int x = 10;} a \texttt{int y = 11;} můžeme změnit pomocí funkce:

\texttt{void fn(int z, int *p)} tím, že nastavíme \texttt{z = 20;} a \texttt{*p = 30;}.
Po zavolání \texttt{fn(x, \&y);} zůstane \texttt{x = 10}, ale \texttt{y = 30} díky použití pointeru.

Další příklad s dynamickou alokací paměti:

\texttt{void fn(int **pa)} provede alokaci paměti pomocí \texttt{*pa = malloc(sizeof(int));}
a následně nastaví hodnotu \texttt{**pa = 10;}.

V \texttt{main()} můžeme použít pointer k tomu, abychom zajistili, že alokovaná paměť nezmizí,
a pointer v \texttt{main()} bude ukazovat na správně alokovanou paměť.

\section{Soubory}

Standardní vstup a výstup jsou v C rozděleny na tři kanály:

\begin{itemize}
    \item \texttt{stdin} - standardní vstup
    \item \texttt{stdout} - standardní výstup
    \item \texttt{stderr} - výstup pro chyby
\end{itemize}

Směrování chybového výstupu v Linuxu:

Přesměrování chybového výstupu do souboru se provádí následovně:

\texttt{příkaz 2> soubor}. Například:

\texttt{ls neexistujici\_soubor 2> error.log}.

Pro přesměrování jak výstupu, tak chybového výstupu do stejného souboru použijeme:

\texttt{příkaz > soubor 2>\&1}. Například:

\texttt{ls neexistujici\_soubor > output.log 2>\&1}.

Pokud chceme oddělit výstup a chyby do různých souborů:

\texttt{příkaz > stdout.log 2> stderr.log}. Například:

\texttt{ls > output.log 2> error.log}.

Chceme-li přidat chybový výstup do existujícího souboru, použijeme append:

\texttt{příkaz 2>> soubor}.

Přesměrování výstupu na \texttt{/dev/null} (zahodit výstup) se provádí pomocí:

\texttt{příkaz 2> /dev/null}.

\section{Řetězení příkazů (pipeline) v Linuxu}

Řetězení příkazů se provádí pomocí znaku \texttt{|}.

Například příkaz:

\texttt{cat txt.txt | my\_prog | grep "ahoj"}

znamená:

\begin{itemize}
    \item \texttt{cat txt.txt} vypíše obsah souboru \texttt{txt.txt} do standardního výstupu.
    \item \texttt{| my\_prog} přesměruje výstup z \texttt{cat txt.txt} do programu \texttt{my\_prog}, který zpracovává text.
    \item \texttt{| grep "ahoj"} filtruje výstup z \texttt{my\_prog} a vypisuje pouze řádky obsahující \texttt{"ahoj"}.
\end{itemize}

Optimalizace: Pokud \texttt{my\_prog} umí číst soubor přímo, můžeme příkaz zjednodušit:

\texttt{my\_prog txt.txt | grep "ahoj"}.

\section{Použití getchar() a putchar()}

Při čtení znaků z vstupu pomocí funkce \texttt{getchar()} můžeme hodnotu vrácenou funkcí přetypovat poté,
co zkontrolujeme, zda se nejedná o \texttt{EOF} (End Of File). Následně můžeme znak předat funkci \texttt{putchar()},
která ho vypíše na výstup.

\texttt{getchar()} čte jeden znak ze standardního vstupu a vrací ho jako hodnotu typu \texttt{int}.
Vrací buď hodnotu znaku, nebo \texttt{EOF}, což je speciální hodnota indikující konec vstupu nebo chybu při čtení.
Z tohoto důvodu je návratový typ \texttt{int}, aby bylo možné odlišit znaky od \texttt{EOF}.

Příklad použití:

\begin{itemize}
    \item Po zavolání \texttt{getchar()} je důležité zkontrolovat, zda vrácená hodnota není \texttt{EOF}.
    \item Pokud je hodnota platná (není \texttt{EOF}), můžeme ji přetypovat na \texttt{char} a následně předat funkci \texttt{putchar()}, která očekává právě \texttt{char}.
\end{itemize}

\section{Práce s čísly}

Pro práci s čísly ve vstupu používáme funkci \texttt{scanf}, například:

\texttt{scanf("\%d", \&x);}, kde proměnná \texttt{x} bude obsahovat načtenou hodnotu.

\section{Práce se soubory}

\texttt{int printf(const char *format, ...)} je funkce, jejíž správné použití zahrnuje například \texttt{printf("\%s", str)}.

\section{Formátovací značky}

Konverze je povinný parametr (\texttt{d}, \texttt{f}, \texttt{s}, \texttt{c}, ...).
Modifikátory, jako \texttt{hh} pro \texttt{short short}, umožňují upřesnit typ dat.
Zarovnání - hodnoty můžeme zarovnat doleva nebo doprava.
Pro výpis speciálních hodnot se používá \texttt{printf} s formátovacími značkami, jako \texttt{\%f} nebo \texttt{\%F}.

Při použití \texttt{scanf("\%s", buffer)} hrozí nebezpečí čtení za hranici alokované paměti.
Pro čtení celého řádku můžeme použít \texttt{int puts()}.

\section{Binarni soubory}

Práce s binárními soubory se principiálně neliší od práce se soubory textovými.
Jazyk C nezná datový typ souboru a nemá syntaktické prostředky pro práci se soubory, proto si pomáháme knihovnami.

\texttt{FILE *} je struktura definovaná ve \texttt{stdio.h}, která popisuje soubor.
\texttt{FILE *fopen} otevírá soubor zadaného jména a vrací \texttt{NULL}, pokud se to nepovede, což je nutné zkontrolovat.
Mod je způsob otevření souboru, zadaný jako řetězec.

\texttt{FILE fclose(FILE *file)} uzavírá soubor a pokud se to nepovede, vrací \texttt{EOF}.
Zápisy se provádějí do bufferu a po nějaké době se zapíší přímo do souboru.
Předčtení souboru pomocí \texttt{abort()} může vést k poškození souboru!

Při práci se soubory také můžeme použít:
\begin{itemize}
    \item \texttt{int getc(FILE *file)} - čte znak ze souboru.
    \item \texttt{int putc(int c, FILE *file)} - zapisuje znak do souboru.
\end{itemize}

Standardní vstupy a výstupy:
\begin{itemize}
    \item \texttt{stdio} - \texttt{FILE *stdin;}
    \item \texttt{FILE *stdout;}
    \item \texttt{FILE *stderr;}
\end{itemize}

Můžeme je použít i pro standardní výstup pomocí \texttt{int fscanf} a \texttt{int fprintf}.

Pro čtení řádku s určením maximální délky můžeme použít \texttt{char *fgets}, což je lepší z hlediska bezpečnosti.
Další užitečné funkce:
\begin{itemize}
    \item \texttt{int ferror(FILE *file)} - zjistí, zda došlo k chybě při práci se souborem.
    \item \texttt{int errno} z knihovny \texttt{<errno.h>} obsahuje poslední chybu I/O operace.
    \item Funkce z \texttt{<string.h>} - \texttt{char *strerror(int errnum)} vrací popis chyby.
\end{itemize}
ftall(f) hodnota kurzoru
fsec(f,
podivat se na video o praci se souborem C
\end{document}
