\documentclass{article}
\usepackage{mathtools, amssymb, amsthm} % imports amsmath

\begin{document}
\sloppy

\section{Strukturovaná analýza a návrh}

Strukturovaný přístup se liší od objektově orientovaného přístupu.
Strukturovaný přístup - systém je považován za kolekci funkcí.
Objektově orientovaný přístup - systém je chápán jako objekty, které mezi sebou komunikují.

\section{Konceptuální model}

Podstata systému, co má systém dělat, co budeme sledovat, nikoli jak to budeme realizovat.

\section{Logický model}

Definuje, jaké jsou požadavky a specifikace systému.

\section{Funkční modelování}

Diagram datových toků (Data Flow Diagram).

\section{Datové modelování}

Cíle:
\begin{itemize}
    \item Chceme mít všechna potřebná data.
    \item A nemít žádná nepotřebná data.
    \item Musíme definovat vztah mezi daty.
    \item Definovat procesy s daty.
\end{itemize}
Sloučit k modelování dat aplikační doménu a jejich vztah "v klidu".

\section{Základní pojmy}

\begin{itemize}
    \item \textbf{Entita} - věc reálného světa neboli objekt (např. banka s ID K666).
    \item \textbf{Entitní množina} - množina entit, které mají stejné vlastnosti.
    Např. entitní množina bank: banka s ID K666.
    \item \textbf{Atribut} - vlastnosti entity (např. banka s ID K666 má atribut jméno poBanka atd. v kontextu našeho problému).
    \item \textbf{Vztah} - asociace mezi několika entitami.
    Např. banka s ID K666 má účet s ID 333 - vztah mezi entitou banky a entitou účet.
    \item \textbf{Vztahová množina} - vztahy stejného typu.
\end{itemize}

\section{Tvorba ER diagramu}

Jednohodnotové a víc hodnotové atributy:
Např. telefon může mít více telefonních čísel.
Prázdné (NULL) atributy mohou představovat chybějící nebo neznámou hodnotu.

\textbf{Odvozené atributy} - hodnotu lze odvodit od jiných atributů nebo entit (např. věk).

Parametry vztahů:
\begin{itemize}
    \item Perzistentní vztahy - např. klient má účet.
    \item Vztahy s průběhem - např. klient zadává objednávku.
\end{itemize}

Jméno vztahové množiny neboli role vlastní nebo vlastník.

Stupeň vztahu:
\begin{itemize}
    \item Uniarni - zaměstnanec -- nadřízený (vzájemně).
    \item Binarni - na konci dvě jiné entity (klient --- účet).
\end{itemize}

Kardinalita - maximální počet vztahů daného typu (typické hodnoty: 1, M).
Členství/účast - minimální počet vztahů (např. účet má minimálně jednoho majitele).

Atributy vztahu - používáme je tehdy, když atribut nemůže být přiřazen k žádné entitě a je tím pádem povýšen na entitu.

ERD diagram není součástí UML.

\section{Pravidla návrhu ER}

\begin{itemize}
    \item Pouze data a jejich vztahy, žádné procesy.
    \item Atribut pouze jednou.
    \item Seskupujeme data pro účely databáze.
    \item Zobrazujeme pouze perzistentní datové objekty.
    \item Zobrazujeme pouze nezbytné vztahy.
\end{itemize}

Pozor na entity, které:
\begin{itemize}
    \item Nemají atributy.
    \item Mají pouze identifikátor.
    \item Mají pouze jeden výskyt.
    \item Obsahují atributy patřící jiným entitám.
\end{itemize}

Jména:
\begin{itemize}
    \item Musí být srozumitelná a vyjadřovat význam entitních a vztahových množin.
    \item Entitní množiny - podstatná jména.
    \item Vztahové množiny - slovesa a předložky.
\end{itemize}

Mezi entitami může být více vztahových množin.

\section{Identifikátor}

Entity a vztahy musí být identifikovatelné.
Hodnota identifikátoru musí být unikátní.

Pravidlo: Pokud je hodnota atributu důležitá, ale neexistuje žádná entita s touto hodnotou, měli bychom ji modelovat jako entitu.

\section{Slabá entitní množina}

Silná entitní množina má identifikátor tvořený vlastními atributy.
Slabá entitní množina nemá identifikátor tvořený vlastními atributy.
Existence slabé entitní množiny závisí na identifikující entitě.

\section{Zobecnění}

Entity mají stejné základy.
Identifikátor entitních množin nižší úrovně je stejný jako u vyšší.

\section{Postup při návrhu ERD}

\begin{itemize}
    \item Zvolte jednu entitu ze specifikace požadavků.
    \item Určete atributy a označte kandidátní klíče.
    \item Prozkoumejte atributy.
\end{itemize}

\end{document}
