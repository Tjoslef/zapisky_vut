\documentclass{article}
\usepackage{mathtools, amssymb, amsthm} % imports amsmath
\begin{document}
\sloppy
\section{Analýza a specifikace požadavků}
\textbf{Plánování testů:}
\begin{itemize}
    \item Sběr informací
    \item Komunikace se zákazníkem, koncovým uživatelem a dalšími zainteresovanými stranami (stakeholdery)
\end{itemize}
Snaha zahrnout všechny strany do procesu, pochopení motivací stakeholderů, hlavně požadavků od zákazníka a důsledků pro zákazníka. Cílem je co nejpřesnější specifikace požadavků zákazníka.

\textbf{Typy požadavků:}
\begin{itemize}
    \item Obchodní požadavky – úspory, čas, atd.
    \item Uživatelé požadavky – co má uživatel se systémem dělat a co má systém dělat
    \item Funkční požadavky – co má být realizováno, aby něco proběhlo (např. diagramy, chování systému za různých podmínek)
    \item Nefunkční požadavky – např. kolik uživatelů může systém obsloužit (kapacita), jaká bude platforma (Android/Linux/Windows), požadavky na bezpečnost, výkon, flexibilitu, spolehlivost (např. zálohování), přátelské rozhraní, měřitelnost požadavků
\end{itemize}

\section{Metody získávání informací}
Snižují riziko, že systém nebude odpovídat požadavkům.

\begin{itemize}
    \item Interview (orientační, strukturované) – běžná a základní forma zjišťování potřeb zákazníka; nejdříve orientační, následně strukturované otázky
    \item Dotazník – velký dosah, problematická užitečnost
    \item Pracovní setkání – skupina stakeholderů vyjednává o požadavcích (menší skupina je lepší)
    \item Sledování procesu na vlastní oči a pochopení – časově náročné
    \item Studium dokumentace
    \item Studium již existujícího systému
\end{itemize}

\section{Problémy při specifikaci požadavků}
\begin{itemize}
    \item Nejasná a neúplná specifikace požadavků
    \item Zákazník nemá jasnou představu, jak by měl celý systém vypadat
    \item Zákazník nedokáže rozhodnout, co je důležité (kvůli nedostatečné znalosti)
    \item Vývojář neorientuje v problematice odvětví (např. finanční sektor)
    \item Prototypování systému – další diskuse se zákazníkem
\end{itemize}
Požadavky je iterativní proces, měly by být co nejvíce specifické a písemně formulované. Je důležité zkoumat, zda jsou požadavky reálné a smysluplné.

\textbf{Validace požadavků:}
Prototypování – částečná implementace, zákazník získává lepší představu. Pokud je prototyp plnohodnotný, lze ho využít v implementaci. Často je však vytvořen narychlo a neměl by se dále používat kvůli nedostatečné kvalitě.

Identifikace požadavků snižuje cenu softwaru.

\section{Dokumentace požadavků}
Kombinace formálních a neformálních technik. Používání vizuálních modelů, tabulek a grafů. Měla by být rozumná v rozsahu.

\textbf{Modelování dat:}
\begin{itemize}
    \item ERD diagramy – strukturovaný model dat
    \item Class diagramy – objektově orientovaný model dat
\end{itemize}

\textbf{Modelování požadavků:}
\begin{itemize}
    \item DFD – specifikace chování systému
    \item UCD – diagram případů užití, doplněný dalšími modely UML
\end{itemize}

\section{Diagram případů užití}
\textbf{Use-case-driven přístup:}
Na co se bude systém používat a kdo ho bude využívat. Diagram musí být dále konkretizován, často za pomoci tabulek. Pokročilé techniky by měly být používány co nejméně, protože mohou být málo srozumitelné.

\section{Diagram aktivit – Prvky}
\begin{itemize}
    \item Uzly
    \begin{itemize}
        \item Akční uzly – modelují aktivitu
        \item Řídicí uzly – modelují rozhodování
    \end{itemize}
\end{itemize}

\end{document}

