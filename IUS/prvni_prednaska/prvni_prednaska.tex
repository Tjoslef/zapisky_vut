\documentclass{article}
\usepackage{mathtools, amssymb, amsthm, graphicx} % imports amsmath
\usepackage[a4paper,top=0.75in, bottom=0.75in,left = 0.75in, right=0.75in]{geometry}
\sloppy
\begin{document}

\section{Úvod}
Specifikace požadavků v UML - diagramy případů užití, diagramy aktivit a stavové diagramy.

Datové modelování - ER diagramy.

Diagramy tříd a diagramy objektů.

Sekvenční diagramy a diagramy komunikace.

Cvičení za 3 body.

\section{Domácí úlohy ER diagram}
Budou také na zkoušce (12 bodů).

\section{Týmový projekt}
Komplexní model informačního systému (16 bodů). Tým 4-5 členů.

\section{Pozor na předmět}
Budou problémy se zkouškou, kde většina studentů měla potíže. Budou v databázovém předmětu – uložit si zadání z domácích úkolů.

\section{Cíl předmětu}
Získat základní přehled ve vývoji SW. Základní modely UML.

\section{Softwarové inženýrství}
Systematický přístup k vývoji, nasazení a údržbě samotného systému. Inženýrská disciplína zabývající se praktickými problémy vývoje SW, zlepšování služeb, snížení lidského faktoru nebo katastrof.

\section{Historie}
Začátek v 50. letech - hlavní náklady na hardware a software nebyly tak drahé. Postupné zlevňování hardwaru, ale náklady na software stoupají. V 60. letech nastala softwarová krize, která vedla k vzniku softwarového inženýrství.

\section{Problémy vývoje softwaru}
Složitost, přizpůsobivost, nestálost, neviditelnost. Syndrom "90\% hotovo" – neschopnost dokončit software.

\section{Některé problémy nemusí projevit vždy}
Práce v týmu je zásadní, každý má jiné vnímání světa, což vede k problémům v komunikaci. Další problémy zahrnují nízkou znovupoužitelnost softwaru, dokumentaci, chyby a stárnutí softwaru.

\section{Specifikace požadavků}
Často nevíme, co zákazník chce. Problémová komunikace mezi obory a nejednoznačný přirozený jazyk.

\section{Rozvoj SW inženýrství}
Výzkum programovacích praktik.

\section{Metodika a její vývoj}
Vývoj komplexních metodik.

\section{Softwarový produkt}
Program je funkční část. Software zahrnuje dokumentaci, program, požadavky... komplexní celek.

\section{Vztah mezi programem a softwarem}
\includegraphics[width=0.8\textwidth]{diagram} % Add relevant image here

\section{Typy softwaru}
Generické: Prodává se libovolnému zájemci (např. krabicový systém).
Zakázkové: Na objednávku zákazníka, cena je výrazně vyšší.

\section{Kvalita SW produktu}
\includegraphics[width=0.8\textwidth]{diagram} % Placeholder for quality diagram

\section{Proces vývoje softwaru}
Proces:
Uživatel zadá požadavky -> Návrh -> Implementace návrhu -> Předání zákazníkovi -> Uživatel.

\section{Životní cyklus softwaru}
Životní cyklus rozděluje proces na fáze, kde každá má svůj cíl. Etapa = etapa životního cyklu softwaru.

\section{Cinnosti spojene s vyvojem softwaru}
Doplnit.

\section{Analýza a specifikace požadavků}
Získávání a analýza uživatelských požadavků, identifikace rizik.

\section{Dekompozice složitých problémů}
Rozdělení složitějších problémů na jednodušší, které jsou lépe zvládnutelné.

\section{Podrobný návrh}
Viz prezentace.

\section{Vodopádový model životního cyklu softwaru}
Požadavek -> Návrh -> Implementace -> Testování -> Provoz, údržba.
Lineární model, často používaný v 70. letech.

\end{document}
