\documentclass{article}
\usepackage{mathtools, amssymb, amsthm,graphicx} % imports amsmath
\begin{document}
\section{uvod}
specifikace pozadavku v UML - diagramy pripady uziti, diagramy aktivit
a stavove diagramy

datove modelovani -ER diagramy!!!!!!!!

Diagramy trid a diagramy objektu

Sekvencni diagramy a diagramy komunikace

cviceni za 3 body
\section{domaci ulohy ER diagram}
budu pote i na zkousce (12bodu)
\section{tymovy projekt}
komplexni model informacniho systemu(16bodu)
tym 4 a 5 clenu
\section{bacha na predmet}
bude problem se zkouskou kde vetsina studentu mela problem
budou pote v databazouvem predmetu ulozit si zadani z domaci ukolu
\section{cil predmetu}
ziskat zakladni prehled ve vyvoji SW
zakladni modely UML
\section{softvero inzenirstvi}
systematicky pristup k vyvoji a nasazeni a take udrzba samotneho systemu
inzenyrska disciplina zabyvajci se praktickymi problemy vyvoji SW
zlepsovani sluzeb a jejiho cenu a celkova produktivita SW
snaah snizeni lidkeho faktoru ci nejake katastrofy
\section{historie}
zacatek 1950 hlavni naklady hardware a software nebyl tak drahy
postupne zlevnovani hardware ale naklady software vstoupaji protoze je komplikovanejsi
a casove je nakladnejsi jelikoz se navazuje na stary softwar(o kterem nikdo moc nevi
a nasledna udrzba rychlejsi zmeny a
60.letech - softwarova krize a nasledny softwarove inzenyrstvi a samotna krize byla
prodluzovani a prodrazovani
spatna kvalita samotneho produktu
nizska produktivita programatora
spatna udrza a inovace produktu
a diky tomu vznikla urcita struktura vyvoje
2krat delsi u 35% projektu
50 az 100 % prekroceni u vetsiny projektu
pouze 7% 100% funkcnost projektu
50-74% funkcnost u  21% projektu
celkkove prumeru 89% vice penez a 2,22% a pouze 61% funkcnost
\section{problemy vyvoje softwaru}
Slozitost - zadny dva softwary nejsou stejne a na to navazuje slozita komunikace v tymu
            problemove pochopit vsechny casti projektu
            nasledne problem s  upravami a rozsirenim
            a tim neschopnost zakaznika presne definovat zadani
Prizpusobivost - hardware a okolnosti se meni a i softwar by se mel menit nejde ale uplne vsechno odhadnout

Nestalost - software musi prezit hardwar prostredku
Neviditelnost - spatna prezentace softwaru a nejde si ho prestavit pred programovani
                a nejsem schopni urcit co v chybi
Syndrom 90% hotovo neschopnost software dokoncit
\section{nemusi projevit vzdy}
prace v tymu zasadni kazdy ma jine vnimani sveta a nasledny problem s komunikaci
- problem organizace a totalni neschopnost planovani
- odchylky znalosti programatoru extrem az 1:20
- nizka znovupouzitelnost pri tvorbe softwaru
        -> malo standartu a casta tvorba od zacatku
        -> maloktery se sestavuje z uz existujicich soucasti
- problem miry
        -> na vetsi projekt jiny system
- problem dokumentace
        -> aktualiza tim jak se software meni
        -> roste a zase problem s jinym vninamim svetu
        -> nemusi odpovidat realite
- nachylnsot softwaru k chybam
-> az uzivatel najde nejake chyby nasledna oprava neni lehka casto
- zpusob starnuti softwaru
-> pridavani novych finkci princip valici se snehove koule chyby se navaluji
\section{starnuti harwaru}
chyby z vyroby nasledna oprava pote klid a s casem se veci opotrebuji pote oprava drazi nez koupit nove auto napr
\section{starnuti softwaru}
neustale zasahovani nova spina zase valici se koule vice bugu atd vetsi a vetsi system
\section{specifikace pozadavku}
casto nevime co vsechno chce zakaznik a samotny predstava zakaznika nebyva predstava
problemova komunikace obor ku oboru
prirozeny jazyk neni jednoznacny
Tvorba softwaru je tvurci proces,software nelze vyrabet
\end{document}
