\documentclass{article}
\usepackage{mathtools, amssymb, amsthm} % imports amsmath
\begin{document}
\sloppy
\section{OPP - objektove orientovany pristupu}
kolekce vzajemne komunikujicihc objektu
vykazuje vyssi stabilitu z pohledu menicich studentu
objektovy navhr nutne neimplikuje objektovou implementaci

objekt reprezentuje entitu reseneho problemu

kazdy objekt ma vymezenou prava a zodpovednost
reprezentuje sam sebe
uchovava data

\section{Abstrakce}
je take ve strukturovane
system objektu je abstarkce reseneho problemu -> zjednodusi pohled na system
a to bez ztraty jeho vyznamu - udaje jsou maji byt releventni
analyza problemu - klasifikace do abstraktnich strukturu
entity klient -> entitni mnozina Klient - strukturovany
objekty klient -> trida Klient
rozpoznavani podobnosti v resene
Tridy objektu
    podobne objekty do stejne tridy
    a trida je seskupeni objektu s podobnymi vlastnostmi
Zapoudreni(ENcapsulation)
    seskupeni souvisejich idei a funkcionalit do jedne jednotku -> moduly atd..
   seskupeni operaci a atributu
Dedicnost(inheritance)
definuje a vytvari(objekty) na zaklade jiz existujicich trid(objektu)
 moznost sdileni chovani bez nutnosti reimplementace
 moznost rozsireni chovani
mezi tridami -> hierarchie podle Dedicnosti
Polymorfismus
    znalost tridy udelani operace ktera je spolecna pro vice trid
    jedna operace muze mit vice operaci
v Domedovy modelu
    je pdobny jako ER diagram ale jsou tam i operace
    vyznacne prvky a vztahy mezi nimi
Jazyk UML
• Unified Modelling Language
• inspirovan existujıcımi analytickymi jazyky a modely

Diagramy chovani,interakce,struktury
Kompozice je silnejsi nez agregace
\end{document}
