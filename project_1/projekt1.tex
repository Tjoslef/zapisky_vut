\documentclass[twocolumn, 10pt, a4paper]{article}
:usepackage[utf8]{inputenc}
\usepackage[T1]{fontenc}
\usepackage{mathtools, amssymb, amsthm,alltt,csquotes} % imports amsmath
\usepackage[left=1.8cm, top=1.8cm, right=1.8cm, bottom=1.8cm, textwidth=18.2cm, textheight=24.1cm]{geometry}
\title{Typografie a publikovani -- 1.projekt}
\author{Petr Veigend}
\author{Petr Veigend \\ \texttt{veigend@fit.vut.cz}}
\begin{document}
\maketitle
\section{Hladka Sazba}
\begin{alltt}
Hladká sazba používá jeden stupeň, druh a řez písma. Sází
se na stránku s pevně stanovenou šířkou. Skládá se z od-
stavců. Odstavec končí východovou řádkou. Věty nesmějí
začínat číslicí.
\end{alltt}
\section{Smíšená sazba}
\begin{alltt}
 Smíšená sazba má o něco volnější pravidla. Klasická hladká
sazba se doplňuje o další řezy písma pro zvýraznění důle-
žitých pojmů. Existuje \uv{pravidlo}:
\end{alltt}
\end{document}
