
\documentclass{article}
\usepackage{mathtools, amssymb, amsthm} % Imports necessary math packages
\usepackage{hyperref} % For clickable links
\usepackage{enumitem} % For custom bullet points
\usepackage{graphicx}% For adding images (if needed)
\usepackage{bookmark}
\usepackage[a4paper,top=0.75in, bottom=0.75in,left = 0.75in, right=0.75in]{geometry}
\sloppy
\title{Prvni-prednaska-uvod}
\author{Tvoje Jméno}
\date{\today}

\begin{document}

\maketitle

\section{Organizace}
\begin{itemize}
    \item Zápočet: 18 bodů
    \begin{itemize}
        \item Laboratoře: 6 bodů
        \item Projekty: 12 bodů (3 body povinné)
    \end{itemize}
    \item Pulsemestrální test: 15 bodů
    \item Rozsah výuky:
    \begin{itemize}
        \item 39 hod přednášky
        \item 6 hod cvičení
        \item 12 hod laboratoře
        \item 8 hod projekty
    \end{itemize}
    \item Zkouška: max 55 bodů, min 27 bodů
\end{itemize}

\section{Úvod}
\begin{itemize}
    \item Studujeme to, abychom věděli, jak pracovat s energií a časem.
    \item Základ pro práci s hardwarem.
    \item Efektivita kódu: počet instalací, doba používání, frekvence vykonávání.
    \item Např.: Baterie a její životnost.
\end{itemize}

\section{Literatura}
\begin{itemize}
    \item Odkaz: \href{http://elektrokinha.cz/}{elektrokinha.cz}
    \item Více IT knih pro elektro.
\end{itemize}

\section{Základní obvod}
\begin{itemize}
    \item Zdroj a spotřebič (mlýn odebírá energii od vody).
    \item Voda teče z vrchu dolů (vyšší potenciál), podobně jako elektrický proud teče z \( + \) do \( - \) (vyšší do nižšího potenciálu).
    \item Napětí je rozdíl potenciálů mezi náboji:
    \[
    W = U \cdot Q \ [J] \quad \text{(práce = napětí krát náboj)}
    \]
    \[
    Q = I \cdot T
    \]
    \[
    W = U \cdot T \cdot I
    \]
    \item Proud prochází žárovkou — zahřívá vlákno, což je práce.
\end{itemize}

\section{Elektrický potenciál v obvodu}
\begin{itemize}
    \item Zdroj vytváří rozdíl potenciálů (= napětí).
    \item Spotřebič odebírá energii částic, které se pohybují, protože jim klade odpor.
    \item Šipka ukazuje směr kopce: z \( + \) do \( - \).
    \item Napětí je opačné ke směru proudu.
\end{itemize}

\section{Důsledky}
\begin{itemize}
    \item Bilance musí být vyrovnaná — Kirchhoffovy zákony!
    \item Podmínka: \textit{Load resistance} (zátěžový odpor) musí být mnohem větší než \textit{Wire resistance} (odpor drátů), aby nesvítily dráty.
    \item Pomocí odporu cíleně řídíme práci a proud.
\end{itemize}

\section{Základní zákony}
\begin{itemize}
    \item Pomáhají navrhovat a řešit obvody.
\end{itemize}

\section{Ohmův zákon}
\begin{itemize}
    \item Závislost mezi napětím a proudem v libovolném úseku obvodu (mezi dvěma body):
    \[
    I = \frac{U}{R}
    \]
    \item Ohmův zákon platí věčně — vodič a spotřebič mívají konstantní odpor, proto závislost napětí a proudu je lineární.
    \item Odpor se však může měnit!
\end{itemize}

\section{Typologie obvodu}
\begin{itemize}
    \item \textbf{Uzel}: místo, kde se stýká více vodičů.
    \item \textbf{Větev}: cesta mezi uzly.
    \item \textbf{Smyčka}: uzavřená cesta tvořená uzlem a dráhami.
\end{itemize}

\section{První Kirchhoffův zákon}
\begin{itemize}
    \item Algebraický součet všech proudů v uzlu je 0. Nábíje nevznikají ani nezanikají v uzlu.
    \item Konvence: proudy do uzlu \( + \), proudy z uzlu \( - \).
\end{itemize}

\section{Druhý Kirchhoffův zákon}
\begin{itemize}
    \item Součet napětí ve smyčce je 0.
    \item Napětí, které vstoupí do smyčky, se musí objevit na spotřebiči.
    \item Zkratovaný obvod má nulové napětí.
\end{itemize}

\section{Ideální zdroj napětí}
\begin{itemize}
    \item Napětí je konstantní, nezávislé na zatížení.
    \item Takový zdroj ale existuje pouze na papíře.
\end{itemize}

\section{Zátěžová charakteristika zdroje}

\begin{itemize}
    \item Čím větší proud, tím menší svorkové napětí (ovlivněno vnitřním\- odporem):
    \[
    I = \frac{U}{R + R}
    \]
    \[
    U = U - R \cdot I
    \]
    \item Vztah:
    \[
    U = \frac{R}{R + R} \cdot U
    \]
\end{itemize}

\section{Vnitřní odpor}
\begin{itemize}
    \item Proměnlivá veličina u článků.
    \item Potenciometr — pomůcka pro pochopení.
\end{itemize}

\section{Zkrat}
\begin{itemize}
    \item Proud: \( I = \frac{U}{R} \)
    \item Práce se koná na zdroji, záleží na vnitřním odporu.
\end{itemize}

\section{Ideální zdroj proudu}
\begin{itemize}
    \item Stále stejný proud bez ohledu na napětí.
    \item Viz graf voltampérové charakteristiky.
\end{itemize}

\section{Dělič napětí}
\begin{itemize}
    \item Obvod, kde jsou dva rezistory připojené v sérii.
\end{itemize}

\end{document}
end{document}
