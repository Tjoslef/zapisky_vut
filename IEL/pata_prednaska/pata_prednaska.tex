\documentclass{article}
\usepackage{mathtools, amssymb, amsthm} % imports amsmath
\begin{document}
\sloppy
\section{polovodice,diody}
odpor vodiveho materialu
merny odpor p je merny odpor
L je delka vodice
S je prurez vodice [m2]
kos(vodic) 10na-8 .. 10na-6
polovodic 10na-6 ... 10na8 - zavisi na cistote materialu
izolant 10na8 az nekonecno
\section{polovodic}
Prvky:Si,Ge,C
Slouceniny:GaAs,SiC,CdS,GaN
organicke Slouceniny

kdyz ho zahrejem zmensi se jeho merny odpor(termistory)
a pro kov je to naopak

merny odpor a taky vodivost je zavisla na primesich -> PN-prechody
pouziti - LED,fotoclanky,lasry,termoclanky,varistory,detektory:plynu
polovodice maji teplotni limity

polovodice v prumeru 4 valecni elektrony
krystalovou mrizku - nepravidelnost muze vest k poruse

atomu v cm na 3 je 10 na 22
pasovy model,sirka zakazaneho pasu 1.1eV
kdyz zahrivame zvysujeme pravdepodobnost ze elektron preskoci z nevodiveho pasu
pres zakazane pasu do vodiveho pasu
vlastni polovodic je vodic je obsahu jenom kremik a je bez primesich
nosic naboje:elektrony a diry(jen fiktivni castive = chybi elektrony
dvoji elektron-diry a rekombinace
role primesi o jednu vice(snadno utrhnuty elektron nebo a jednu mene(dira)

dva typy primesich - P-> akceptory diry
                     N-> donory 5 valacnych elektrony
 100x pridam primes 100x mensi merny odpor
https://youtu.be/Fwj_d3uO5g8?si=d8S_WE6ZDtyTY_an
na zaklade difuze vznika potencialova bariera
    propustny smer(Forward)
        napetu ma zapornou teplotni zavislot - teplomer
    zaverny smer
        zaverny proud velm roste s teplotou
        prurazne napeti prechodu Ubr - lavinovy pruraz
kapacita prechodu C zvysuje se Ur
doba zotaveni T_{rr} - nikdy neni nulovy naboj musi pryc z PN prechodu

\section{zavislost na U}
I = I_o (e na qu/kT -1)

q je naboj elektronu
k je Boltzmanova konstanta
T je teplota prechodu PN (300K)

vzorecek vhodny spise pro propustny smer

\end{document}
