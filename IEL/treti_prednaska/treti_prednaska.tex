\documentclass{article}
\usepackage{mathtools, amssymb, amsthm} % imports amsmath
\begin{document}
\sloppy
\title{Kondenzator a civka,kapacita a indukcnost}
\section{pasivni prvky el obvodu}
rezistor - soucaska s specifickym odporem
R = p(l/s) delsi vetsi odpor a u sirky ci mensi tim vetsi odpor
I = U/R
zalezi na materialu
omezeni proudu v nejake vetvi -> konkretni ubytek napeti
vyvazovani napeti
\section{Kondenzator}
elektrostaticke pole - prostredi ve kterem jsou el naboje v klidu -> netecte tam proudu
vrstvy -> vodic - dielektrikum(izolant) - vodic -> kdyz neco nejde vem si vetsi kladivo
\section{Coulombuv zakon}
F = k(Q*Q)/r2 - r vzadelonst
k = 1/(4\pi* \e)
symbol e znaci permitivitu
e0 je permitivita vakua
e = e0*er
elektrostaticke je vyuzivano v laserova tiskarne
\section{kondezator}
Q = CU - napeti mezi deskama
kapacita -> C = er* e0* (S/l) cim mensi l delka tim lepsi izolant jelikoz
jsou blize
cim vetsi plocha s tim vetsi kapacita
- vlastnosti materialu
- je mysleny jako zasobni el naboje
kapacita fyzilani velicina a kondezator obvodovy prvek
pomoci kondezatoru umistujeme v obvodu kapcitu kde ji potrebujeme
- ale kapacita existuje vsude
kondezator - dva vyvody
C - velicina Farad
provedeni kondezatoru keramicky(jednodusi a levnejsi)
hlinikovy elektrolyticky(vetsi kapacita)
\section{spojovani kondenzatoru}
vyssi kapacita -> paralerni propojeni jeliko U je vetsi nez aby bylo zapojeny seriove
v serii se scitaji prevracene hodnoty -> vyzivame kdyz mame velke napeti -> maji omezeni napeti
diky nemu predstava byt proud staticky tece jenom poku se kondezator se vybiji a nabiji
i kdyz netece proud je v kondezatoru napeti a je v nem urciti proud
kdyz ho nabijem a ptom ho zapojime do obvodu s pouze s rezistorem atd zacne pusobit jako zdroj
pouziva se vykryti nouze nez dorazi zdroj
    typicky rychle skokove zmeny stavu cislivovych obvodu
    vyhlazeni napeti ze stridaveho zdroje
setrvacni prvek
     prod definoane reakce obvodu
     nastaveni frekvence kmitani obvodu
     odlozeni reakce obvodu na nejakou zmenu stavu
section{energie nabitheo kondezatoru}
w = (1/2)CU2
priklad defibrilator
C = 100 uF
U = 4000V
t = 2ms
a = 0.25    W = (1/2)*CU2 = 800j

\end{document}
