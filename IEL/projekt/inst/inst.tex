\documentclass{article}
\usepackage{mathtools, amssymb, amsthm} % imports amsmath
\begin{document}
\sloppy
\[
    U_{Rb} = I_{Rb4} \times R_b
\]
\[
    U_{R4} = I_{Rb4} \times R_4
\]
\[
    U_{Rc} = I_{Rc5} \times R_c
\]
\[
    U_{R5} = I_{Rc5} \times R_5
\]

% Zpětný dopočet napětí a proudu z hvězda -> trojúhelník:
% II. Kirchhoffův zákon:
\[
    U_{R2} = U - U_{R5}
\]
\[
    U_{R1} = U - U_{R4}
\]
\[
    U_{R3} = U_{R2} - U_{R1}
\]

% Ohmův zákon:
\[
    I_{R1} = \frac{U_{R1}}{R_1}
\]
\[
    I_{R2} = \frac{U_{R2}}{R_2}
\]
\[
    I_{R3} = \frac{U_{R3}}{R_3}
\]

\[
    I_{R4} = I_{Rb4}
\]
\[
    I_{R5} = I_{Rc5}
\]

% Kontrola I. Kirchhoffův zákon:
\[
    I_{R1} = I_{R3} + I_{R4}
\]
% Nebo také:
\[
    \lvert I_{R1} - (I_{R3} + I_{R4}) \rvert < \epsilon
\]
\[
    I_{R2} + I_{R3} = I_{R5}
\]
% Nebo také:
\[
    \lvert (I_{R2} + I_{R3}) - I_{R5} \rvert < \epsilon
\]
\end{document}

