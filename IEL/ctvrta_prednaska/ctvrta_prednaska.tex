\documentclass{article}
\usepackage{mathtools, amssymb, amsthm} % imports amsmath
\begin{document}
\sloppy
\section{Prechodove jevy RC,RL,RLC}
prechodovy jev = prechod z jednoho ustaleneho stavu do jinehe
je zpusobena skokovou zmenou napeti na vstupu
RC clanek - integracni a derivacni
v roli parazatini kapacit
reseni nezatizeneho RC clanku - domaci ukol zatizeny
\section{integracni}
i(t) = C(du_2(t)/dt) u_2 = 0 pocatecni hodnota!!!
uR(t) = R*i(t)
u1(t) = ur (t) + u_2(t)
u_2(t) = u_1(1-e -t/RC) = 1 -e -t/RC
po 5 * tau obvod se dostane na plne napeti
1 tau = 63(/'%'/)
na zkousku lol
pro vybijeni je prubeh stejny jako pro nabijeni po 1 tau 37%
\section{derivacni RC clanek}
u_2 je napeti na rezistoru
napeti skoci okamzite na 1 na rezistoru a potom se zacne se kondenzator nabije
a napeti zacne na rezistoru klesat a kondenzator se nabije a napeti na rezistoru = 0
po jednou tau je na rezistoru 37 procent a kondenzator je nabiti na 63 procent
vybijeni napeti na rezistoru -1 postupne stoupa do 0 postupne

tau = RC
domaci ukol RC clanek zatizeny odporem R_c
zhorsuji vykon cipu
\section{RL clanek}
casova hodnota R*L ale L/R nejvetsi rozdil mezi RC a RL
\section{RLC}
seriovy RLC obvod
system druhe radu - existuje vice moznych tvaru reseni
netlumeni R = 0
male tlumeni - ustalene male kmitani
kriticka mez tlumeni - neprekmitne, nejrychleji dosahne rovnovazneho stavu
RLC obvod muze pozit jako nahradni zapojeni kondenzatoru
realne soucastky vypadaji jako RLC -> parazatini indukcnost a parazatini odpory
/section{shrnuti}
RC se pouzivaji blokovani ,vazebni,casovaci obvody
preslechy(crosstalk) -> stineni a diferencialni signaly
\end{document}
