\documentclass{article}
\usepackage{mathtools, amssymb, amsthm} % imports amsmath
\begin{document}
\sloppy
\section{NPN prechod}
jsou to dve diody ale nejsou \smile
Emitor Baze Kolektor
kdybychom nepripojili Bazi tak by byl NPN prechod zavreny a nepruchodny
kdyz pripojime z bazi kladne tak budeme odsavat elektrony ktery by skocili do der
v bazi -> proud bude prochazet protoze v baze neni zaporne nabita  -> proudem v jedne smyce
->ovladame proud v druhe smycce v mensi smyce muze byt proud daleko slabsi
zesileni \beta = I_k / I_B
u emitoru byva sipka
npn sipka pryc a pnp dovnitr
\section{Bipolarni tranzistor}
misto odsavani tak dodavam diry
emitor a kolektor maji jine vlastnosti
vstupni charakteristika baze-emitor
  -> chovase jak dioda
vystupni pri U_{BE}

charakteristika je vazana proudem na Bazi
kolektor a emitor musi byt dostatecne napeti
z leva kolektor uprostred base a vpravo emitor
jsou velice zavisle na teplotu jelikoz zacina se projevovat vodivost materialu
kterou nechceme a zacneme se chovat jinak
zesileni transitoru je velice zavisli na presnosti vyroby transitoru
tranzistor jako spinac
napeti na mezi basi a emitorem byt vetsi nez 0.5 V a do saturace 0.7
tr
\end{document}
